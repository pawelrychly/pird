
\documentclass{article} 
\usepackage{polski} %moze wymagac dokonfigurowania latexa, ale jest lepszy ni� standardowy babel'owy [polish] 
\usepackage[cp1250]{inputenc} 
\usepackage[OT4]{fontenc} 
\usepackage{float}
\usepackage{graphicx,color} %include pdf's (and png's for raster graphics... avoid raster graphics!) 
\usepackage{url} 
\usepackage{listings}
\usepackage[pdftex,hyperfootnotes=false,pdfborder={0 0 0}]{hyperref} %za wszystkimi pakietami; pdfborder nie wszedzie tak samo zaimplementowane bo specyfikacja nieprecyzyjna; pod miktex'em po prostu nie widac wtedy ramek


\input{_ustawienia.tex}

\title{Opis projektu:\\Przetwarzanie i Rozpoznawanie D�wi�ku}
\author{Pawe� Rych�y}
\date{}


\begin{document}

\thispagestyle{empty} %bez numeru strony

\begin{center}
{\large{Przetwarzanie i Rozpoznawanie Dzwi�ku}}

\vspace{3ex}

Klasyfikacja utwor�w muzycznych ze zbioru gtzan.


\vspace{3ex}

{\footnotesize\today}

\end{center}


\vspace{10ex}

\vspace{5ex}

Autor:
\begin{tabular}{lllr}
\textbf{Pawe� Rych�y} & inf94362 & ISWD & pawelrychly@gmail.com \\
\textbf{Dawid Wi�niewski} & inf94387 & ISWD & wisniewski.dawid@gmail.com \\
\end{tabular}

\vspace{5ex}

\newpage





\section{Cel projektu}




%%%%%%%%%%%%%%%% literatura %%%%%%%%%%%%%%%%

\bibliography{sprawozd}
\bibliographystyle{plain}


\end{document}

